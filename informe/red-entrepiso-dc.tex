\subsubsection{Descripción y grafo de relación entre los nodos}

Realizamos una captura en la red Wi-Fi \emph{Entrepiso-DC}, disponible desde los laboratorios del Depto. de Computación. La muestra fue tomada un lunes a las 17hs aproximadamente -horario típicamente de alto tráfico-, logrando un total de 450 paquetes en 11 minutos.

En el gráfico de la figura \ref{fig:entrepiso-dc-grafo} se presenta el grafo dirigido representando la red. En el mismo se observa una gran cantidad de nodos ligada al nodo con IP \emph{10.1.200.197}, y luego múltiples conjuntos pequeños de nodos conectados entre sí pero disconexos de la estructura mayoritaria. Es decir, a diferencia de las demás redes observadas, esta presentó una gran cantidad de pequeños clusters independientes.

\begin{figure}[H]
  \begin{center}
    \includegraphics[width=0.6\linewidth]{../imgs/red-entrepiso-dc_red.png}
    \caption{Grafo mostrando la topología de la red \emph{Entrepiso-DC}.}
    \label{fig:entrepiso-dc-grafo}
  \end{center}
\end{figure}

\subsubsection{Fuente: $S_{dst}$}

\begin{figure}[H]\centering
    \includegraphics[width=\linewidth]{../imgs/red-entrepiso-dc_S_dst_hist.png}
    \caption{Histograma de $S_{dst}$}\label{fig:entrepiso-dc-dst-hist}
\end{figure}

\begin{figure}[H]\centering
    \includegraphics[width=\linewidth]{../imgs/red-entrepiso-dc_S_dst_info.png}
    \caption{Informacion de $S_{dst}$}\label{fig:entrepiso-dc-dst-info}
\end{figure}

$\bullet$ Entropía de la fuente: 6.95 sobre un máximo posible de 7.58

En las figuras \ref{fig:entrepiso-dc-dst-hist} y \ref{fig:entrepiso-dc-dst-info} se presentan los gráficos tipo histograma e información para la fuente $S_{dst}$. En este caso, hay múltiples nodos con una frecuencia mucho mayor que la media, siendo distinguidos aproximadamente 10 nodos según el criterio de la sección \ref{subsec:modelos-fuente-informacion}.

\subsubsection{Fuente: $S_{src}$}

\begin{figure}[H]\centering
    \includegraphics[width=\linewidth]{../imgs/red-entrepiso-dc_S_src_hist.png}
    \caption{Histograma de $S_{src}$}\label{fig:entrepiso-dc-src-hist}
\end{figure}

\begin{figure}[H]\centering
    \includegraphics[width=\linewidth]{../imgs/red-entrepiso-dc_S_src_info.png}
    \caption{Informacion de $S_{src}$}\label{fig:entrepiso-dc-src-info}
\end{figure}

$\bullet$ Entropía de la fuente: 2.03 sobre un máximo posible de 5.00

En las figuras \ref{fig:entrepiso-dc-src-hist} y \ref{fig:entrepiso-dc-src-info} se presentan los gráficos tipo histograma e información para la fuente $S_{src}$. A diferencia de la fuente $S_{dst}$, en este caso hay un único nodo distinguido, el de IP \emph{10.1.200.197}.

\subsubsection{Discusión}

En el caso de esta red, la entropía de la fuente $S_{dst}$ es mucho más alta en términos relativos (92\% de la máxima posible) que la de la fuente $S_{src}$ (41\%). Esto es previsible dados los gráficos de información, dado que $S_{src}$ tiene una estructura más jerarquizada respecto al nodo distinguido; complementándose con lo que se observa en el gráfico del grafo.
Enfocando el análisis el cluster que aglomera la mayor cantidad de nodos, la IP \emph{10.1.200.197} es la que envía y recibe mayor cantidad de paquetes, y por lo tanto pareciera tratarse del gateway.

Suponemos que son pequeños grupos de terminales que se comunican entre sí, pero no con el conjunto general (que suponemos que lleva a Internet). Una posibilidad es que sean máquinas de oficina que accedan entre si, o a una terminal que da un servicio (ej. impresora).